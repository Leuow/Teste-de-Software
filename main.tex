\documentclass[12pt]{article}%
\usepackage{amsfonts}
\usepackage{fancyhdr}
\usepackage{comment}
\usepackage[a4paper, top=2.5cm, bottom=2.5cm, left=2.2cm, right=2.2cm]%
{geometry}
\usepackage{times}
\usepackage{amsmath}
\usepackage{changepage}
\usepackage{amssymb}
\usepackage{graphicx}%
\usepackage[brazilian]{babel}
\usepackage[utf8]{inputenc}
\usepackage[T1]{fontenc}
\newenvironment{proof}[1][Proof]{\textbf{#1.} }{\ \rule{0.5em}{0.5em}}


\begin{document}

\title{Teste de Software}
\author{Leonardo Maia de Lima}
\maketitle
\section{Teste de Caixa Branca ou Funcional}

O teste funcional, ou de caixa-preta, é baseado nos requisitos funcionais do software. Esta técnica não está preocupada com o comportamento interno do sistema durante a execução do teste, mas sim com a saída gerada após a entrada dos dados especificados. Tal tipo de teste é indicado para detectar erros de interface, de comportamento e/ou desempenho, podendo ser aplicada em todas as fases de testes (unidade, integração, sistema e aceitação). Uma dificuldade dessa técnina, por questões de tempo e recurso, é testar todas as entradas possíveis.
Também chamada de teste funcional, teste comportamental, orientado a dado ou orientado a entrada e saída, a técnica de caixa-preta avalia o comportamento externo do componente de software, sem se considerar o comportamento interno do mesmo.\\ Dados de entrada são fornecidos, o teste é executado e o resultado obtido é comparado a um resultado esperado previamente conhecido. Como detalhes de implementação não são considerados, os casos de teste são todos derivados da especificação.
 

\subsection{Particionamento em classes de equivalência}
Esse tipo de teste visa otimizar os casos de teste fazendo a maior cobertura possível do sistema, eliminando os casos redundantes. Para a execução desta técnica, seguimos algumas etapas.
\subsubsection{Identificar as Classes de Equivalência}
– Identificar, através da especificação dos dados de entrada, as classes de equivalência \\
– Caracterizar pelo menos 2 grupos de valores de entrada para cada classe: \\
\indent • Válidas: Valores Esperados pelo programa \\
\indent • Inválidas: Entradas Inválidas ou inesperadas


\subsection{Refinamento das Classes de Entrada}
– Identificar se há razão para acreditar que elementos de uma mesma classe são tratados de forma diferente pelo
programa, se sim, dividir estas em classes menores \\
– Particionar Classes cujos atributos são considerados elementos chave para o item em teste.
\subsection{Identificar as Classes de Saída}
– Mapear todas as possíveis classes de saídas geradas pelo item em teste

\subsection{Refinar as Classes de Saída}
– Verificar se alguma classe de saída merece ser subdividida em classes menores
• Ex: Código Genérico de Erro pode ser desmembrado em outros códigos.
\subsection{Relacionar Classes de Entrada e Classes de Saída}
– Criar um mapeamento entre as classes de entrada e as classe de saída esperadas
\subsection{Derivar os Casos de Teste }
– Atribuir um valor de entrada para cada classe de entrada (válidas e inválidas) \\
– Elaborar casos de teste para as combinações das classes válidas \\
– Elaborar casos de teste para todas as classes inválidas \\
– Derivar casos de teste que cubram, de uma só vez, o maior número de classes válidas/inválidas



\section{Análise do Valor Limite}
É um complemento do particionamento de classes, onde são gerados casos focados nos limites das partições (valores mínimos e máximos)\\
Exemplo: Os valores de entrada só podem estar no intervalo entre 1900 e 2018, então devemos testar os valores 1899, 1900, 2018 e 2019. (Não testamos os valores 1901 e 2017 pois estão dentro do intervalo de valores válidos).

\section{Example 3. The whole is equal to the sum of the parts}

The sum of two numbers

\subsection{Solution}
In tis problem, we are asked to find two numbers. Therefore, we must let x be one
    
\section{Example 4}     

The sum of two consecutive numbers is 37. What are they?

\subsection{Solution}

Two consecutive numbers are 8 and 9, or 51 and 52. Let x, then, be the first  


The two numbers are 18 and 19. 

\section{Example 5}

One number is 10 more than anot

\subsection{Solution}

Let x be the smaller number.



That's the smaller number.The larger number is 10 more:15 

\section{Example 6} 

Divide \$80 among three people so that the second will have twice as much as the first, and the third will have \$5 less than the second.

\subsection{Solution}

Again we are asked to find one more number. We must begin by letting x
  
  
 \bibliography{bibliografia}

\end{document}